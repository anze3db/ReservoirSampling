\documentclass[journal]{IEEEtran/IEEEtran}

\usepackage[utf8x]{inputenc}
\usepackage{cite}
\usepackage{hyperref}
\usepackage{url}
\usepackage{verbatim}
\usepackage{epstopdf}
\usepackage[pdftex]{graphicx}
\graphicspath{{img/}}
\DeclareGraphicsExtensions{.eps,.pdf,.png,.jpg,.jpeg}
\usepackage[cmex10]{amsmath}
\usepackage{array}

\usepackage{fancyvrb}
\DefineVerbatimEnvironment{code}{Verbatim}{fontsize=\small}


% correct bad hyphenation here
\hyphenation{op-tical net-works semi-conduc-tor}

\begin{document}
\title{Reservoir Sampling}

\author{Anže~Pečar, Miha~Zidar}%
%\thanks{M. Shell is with the Department
%of Electrical and Computer Engineering, Georgia Institute of Technology, Atlanta,
%GA, 30332 USA e-mail: (see http://www.michaelshell.org/contact.html).}% <-this % stops a space
%\thanks{J. Doe and J. Doe are with Anonymous University.}% <-this % stops a space
%\thanks{Manuscript received April 19, 2005; revised January 11, 2007.}}

\markboth{MLDM Workshop, January~2012}{}
%\IEEEspecialpapernotice{(Invited Paper)}

\maketitle

\begin{abstract}
The abstract goes here.[TODO]
\end{abstract}

\begin{IEEEkeywords}
[TODO]
\end{IEEEkeywords}

\IEEEpeerreviewmaketitle


\section{Introduction}
\IEEEPARstart{T}{he} problem with random sampling is to select a random sample of size $n$ from a set of size $N$. Many algorithms have been developed for this problem when the value of $N$ is known beforehand. Some problems, that we encounter in the real world, do not have a specified $N$ or $N$ cannot be determined efficiently. Those kind of problems will be the focus of this paper. We shall take a look at older algorithms such as Algorithm R which was developed to efficiently and accurately create a random sample from a tape in one pass, as well as newer algorithms, such as [TODO]\\

Many reservoir algorithms make the assumption that the data does not change over time. Alogrithm R, for an example, samples starting data with a greater frequency than the data at the end of the tape, which is efficient but it fails to react to emarging trends in the data. If we are interested in trends in our data a different kind of algorithm needs to be used. [TODO]\\\\

%\hfill mds%?
 
\hfill \today
\section{Algorithm R and its improvements}
\subsection{Motivation}
In order to better understand the complex algorithms in use today we first need to understand the idea behind simpler algorithms used in the past. In this section we shall describe Algorithm R and its improvement - Algorithm Z.

\subsection{Algorithm R}
Algoritm R is a reservoir algorithm written by Alan Waterman. It works as follows: we fill up our reservoir of $n$ records with the first $n$ records of the file we are processing. The $t + 1$st record then has a $n/(t + 1)$ chance of being in our reservoir of size $n$. The candidate it replaces is chosen randomly from the $n$ candidates.
\\\\
Below is t implementation of Algorithm R in $Python$:\\
\begin{code}
from random import randrange

N = [0] * n # initial reservoir
for i in xrange(n):
	N[i] = READ_NEXT_RECORD()
t = n
while EXISTS_NEXT_RECORD():
	t += 1
	M = randrange(0, t, 1)
	if M < n:
		N[M] = READ_NEXT_RECORD()
	else:
		READ_NEXT_RECORD()	
\end{code}
Explanation:\\
The list $N$ is our reservoir in which we store a random sample of records. We start of with an empty list of length $n$ in which we store the first $n$ elements (the for loop). The function $READ\_NEXT\_RECORD()$ returns the next record in the stream. After we have inserted the first $n$ elements into our reservoir, we enter the $while$ clause, which runs as long as there are still records in the stream. In each run of the $while$ clause we first increment the counter of records $t$ and then calculate a random number between $0$ and $t$. We store the random number in $M$. If $M$ is lower than $n$ we store the record at index $M$, otherwise we skip the record. [AM I EXPLAINING THE OBVIOUS?]
\subsection{Algorithm Z}
It turns out we do not need to go over every single record in order to get a random set in the reservoir. We can skip a random number of records each time just as long as the probability of a record being in the reservoir does not change. This is the idea behind Algorithm Z.\\ \\
Below is the first part of Algorithm Z implemented in $Python$:\\
\begin{code}
from random import randrange

N = [0] * n # initial reservoir
for i in xrange(n):
	N[i] = READ_NEXT_RECORD()
t = n
thresh = T * n
num = 0
while EXISTS_NEXT_RECORD() and t <= thresh:
	t += 1
	num += 1
	V = random()
	S = 0
	quot = num/t
	while quot > V:
		S += 1
		t += 1
		num += 1
		quot = (quot * num)/t
	
	SKIP_RECORDS(S)
	
	if EXISTS_NEXT_RECORD():
		M = randrange(0, n)
		N[M] = READ_NEXT_RECORD()
\end{code}
The algorithm works best if $T$ is between $10$ and $40$. When the $thresh$ value gets large enough the algorithm starts using the rejection technique. The rejection technique generates $S$ even more slowly than in the above algorithm, which makes the algorithm faster as there are even more records that we are able to skip. The main difference between $Algorithm\ R$ and $Algorithm\ Z$ is that the former  computes every single record while the latter skips as many records as possible.  
\section{Online learning}
[TODO]


% Note that IEEE typically puts floats only at the top, even when this
% results in a large percentage of a column being occupied by floats.
%\begin{figure}[!t]
%\centering
%\includegraphics[width=2.5in]{myfigure}
% where an .eps filename suffix will be assumed under latex, 
% and a .pdf suffix will be assumed for pdflatex; or what has been declared
% via \DeclareGraphicsExtensions.
%\caption{Simulation Results}
%\label{fig_sim}
%\end{figure}

% An example of a double column floating figure using two subfigures.
% (The subfig.sty package must be loaded for this to work.)
% The subfigure \label commands are set within each subfloat command, the
% \label for the overall figure must come after \caption.
% \hfil must be used as a separator to get equal spacing.
% The subfigure.sty package works much the same way, except \subfigure is
% used instead of \subfloat.
%
%\begin{figure*}[!t]
%\centerline{\subfloat[Case I]\includegraphics[width=2.5in]{subfigcase1}%
%\label{fig_first_case}}
%\hfil
%\subfloat[Case II]{\includegraphics[width=2.5in]{subfigcase2}%
%\label{fig_second_case}}}
%\caption{Simulation results}
%\label{fig_sim}
%\end{figure*}
%
% Note that often IEEE papers with subfigures do not employ subfigure
% captions (using the optional argument to \subfloat), but instead will
% reference/describe all of them (a), (b), etc., within the main caption.


% An example of a floating table. Note that, for IEEE style tables, the 
% \caption command should come BEFORE the table. Table text will default to
% \footnotesize as IEEE normally uses this smaller font for tables.
% The \label must come after \caption as always.
%
%\begin{table}[!t]
%% increase table row spacing, adjust to taste
%\renewcommand{\arraystretch}{1.3}
% if using array.sty, it might be a good idea to tweak the value of
% \extrarowheight as needed to properly center the text within the cells
%\caption{An Example of a Table}
%\label{table_example}
%\centering
%% Some packages, such as MDW tools, offer better commands for making tables
%% than the plain LaTeX2e tabular which is used here.
%\begin{tabular}{|c||c|}
%\hline
%One & Two\\
%\hline
%Three & Four\\
%\hline
%\end{tabular}
%\end{table}

\section{Conclusion}
The conclusion goes here.


% if have a single appendix:
%\appendix[Proof of the Zonklar Equations]
% or
%\appendix  % for no appendix heading
% do not use \section anymore after \appendix, only \section*
% is possibly needed

\appendices
\section{Appendix title}
Appendix one text goes here.

\section{}
Appendix two text goes here.


\section*{Acknowledgment}


The authors would like to thank...

% trigger a \newpage just before the given reference
% number - used to balance the columns on the last page
% adjust value as needed - may need to be readjusted if
% the document is modified later
%\IEEEtriggeratref{8}
% The "triggered" command can be changed if desired:
%\IEEEtriggercmd{\enlargethispage{-5in}}

% references section

% can use a bibliography generated by BibTeX as a .bbl file
% BibTeX documentation can be easily obtained at:
% http://www.ctan.org/tex-archive/biblio/bibtex/contrib/doc/
% The IEEEtran BibTeX style support page is at:
% http://www.michaelshell.org/tex/ieeetran/bibtex/
%\bibliographystyle{IEEEtran}
% argument is your BibTeX string definitions and bibliography database(s)
%\bibliography{IEEEabrv,../bib/paper}
%
% <OR> manually copy in the resultant .bbl file
% set second argument of \begin to the number of references
% (used to reserve space for the reference number labels box)
\begin{thebibliography}{1}

\bibitem{IEEEhowto:kopka}
H.~Kopka and P.~W. Daly, \emph{A Guide to \LaTeX}, 3rd~ed.\hskip 1em plus
  0.5em minus 0.4em\relax Harlow, England: Addison-Wesley, 1999.
\end{thebibliography}

% biography section
% 
% If you have an EPS/PDF photo (graphicx package needed) extra braces are
% needed around the contents of the optional argument to biography to prevent
% the LaTeX parser from getting confused when it sees the complicated
% \includegraphics command within an optional argument. (You could create
% your own custom macro containing the \includegraphics command to make things
% simpler here.)
%\begin{biography}[{\includegraphics[width=1in,height=1.25in,clip,keepaspectratio]{mshell}}]{Michael Shell}
% or if you just want to reserve a space for a photo:
\newpage

\begin{IEEEbiography}[{\includegraphics[width=1in,height=1.25in,clip,keepaspectratio]{anze}}]{Anže Pečar}
Biography text here.
\end{IEEEbiography}

% if you will not have a photo at all:
\begin{IEEEbiography}{Miha Zidar}
Biography text here.
\end{IEEEbiography}

% You can push biographies down or up by placing
% a \vfill before or after them. The appropriate
% use of \vfill depends on what kind of text is
% on the last page and whether or not the columns
% are being equalized.

%\vfill

% Can be used to pull up biographies so that the bottom of the last one
% is flush with the other column.
%\enlargethispage{-5in}

\end{document}


